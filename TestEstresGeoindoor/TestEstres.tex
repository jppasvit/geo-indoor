\documentclass[10pt,a4paper]{article}
\usepackage[latin1]{inputenc}
\usepackage{amsmath}
\usepackage{amsfonts}
\usepackage{amssymb}
\usepackage{graphicx}
\usepackage{hyperref}
\author{Juan Pedro Pascual Vitores}
\title{Test de estr�s Geoindoor}
\begin{document}
\maketitle
Debido a que Geoindoor funciona con servicios web como Heroku y Firebase, las pruebas de estr�s y rendimiento est�n limitadas por las caracter�sticas de ambos servicios.
\section{Heroku}

Heroku es utilizado para implementar un servidor que nos proporcione una API rest, las limitaciones de heroku son las siguientes.

\subsection{HTTP timeouts} 

Las solicitudes HTTP tienen tiempo de espera de 30 segundos en la que el proceso web debe devolver los datos de respuesta. Los procesos que no env�an datos de respuesta dentro de 30 segundos ver�n un error H12 en sus registros.

Despu�s de la respuesta inicial, cada byte enviado desde el servidor reinicia su timeout a 55 segundos.

Si no se reciben datos del dyno\footnote{Todas las aplicaciones de Heroku se ejecutan en una colecci�n de contenedores de Linux llamados dynos} dentro de 55 segundos, la conexi�n finaliza y se registra un error H15.

De igual manera, si no se reciben datos del cliente dentro de 55 segundos, la conexi�n finaliza y se registra un error H28.

\subsection{Buffer de respuesta HTTP}

Se mantiene un buffer de 1MB para las respuestas del dyno por conexi�n.

\subsection{Buffer de solicitud HTTP}

Al procesar una solicitud entrante, se configura un b�fer de recepci�n de 8 KB y se comienza a leer la solicitud y se solicita encabezados. Cada uno de estos puede tener un m�ximo de 8 KB de longitud, pero juntos pueden ser de m�s de 8 KB en total. Las solicitudes que contienen una l�nea de solicitud o una l�nea de encabezado de m�s de 8 KB se eliminar�n.

\subsection{Conexiones simult�neas}

herokuapp.com permite muchas conexiones concurrentes a dynos web.

\subsection{Dynos}

Las versi�n free tiene 512MB de RAM

\section{Firebase}
Aqu� todo se refiere a la funcionalidad de Firebase que estamos utilizando, Realtime Database.
\\
- Conexiones simult�neas: 100,000 // 100 en free
\\
- Capacidad: 1GB en free
\\
- Respuestas simult�neas enviadas desde una misma base de datos: 100,000/s
\\
- Tama�o de escritura de la base de datos que activa una funci�n: 1 MB



\subsection{�rbol de datos}
- Profundidad m�xima de los nodos secundarios: 32
\\
- Longitud de una clave: 768 bytes
\\
- Tama�o m�ximo de un String: 10 MB

\subsection{Lecturas}
- Tama�o de una sola respuesta atendida por la base de datos: 256 MB
\\
- Cantidad total de nodos en una ruta que tiene agentes de escucha o consultas: 75 millones 

\subsection{Escrituras}
- Tama�o de una �nica solicitud de escritura en la base de datos: 16 MB
\\
- Bytes escritos: 64 MB/minuto


\begin{thebibliography}{a}

\bibitem{Heroku}
\textit{Heroku Limits}
[ Fecha de consulta: 25/11/2017 ]
\url{https://devcenter.heroku.com/articles/limits}

\bibitem{Firebase}
\textit{Firebase Limits}
[ Fecha de consulta: 25/11/2017 ]
\url{https://firebase.google.com/docs/database/usage/limits?hl=es-419}


\end{thebibliography}

\end{document}