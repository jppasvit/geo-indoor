\capitulo{1}{Introducción}

\texttt{"No se dónde está"}\\
\\
Es habitual escuchar esta frase, que multitud de veces irrumpe en nuestras vidas dificultando nuestras tareas, tareas
tan simples como encontrar un baño, una oficina, o la cafetería. Éstos son los típicos problemas que se encuentra uno 
cuando acude por primera vez a un lugar, pero es más grave para aquel que ofrece un servicio o unos productos en un 
determinado lugar y el usuario no lo localiza. Ésto puede hacer que el flujo de personas que acudan al establecimiento 
sea mucho menor, lo que provocaría que el servicio estuviera poco solicitado o que los productos ofrecidos no fuesen vistos 
por tantas personas como se desearía. Es aún más grave si hablamos de un negocio el cual necesite que los usuarios localicen
con rapidez donde se ofrecen los servicios o productos. 
 
De ahí la necesidad de la implementación del proyecto, que propone dar servicios de geolocalización indoor. Ésto resolvería 
diversos problemas en establecimientos como museos, exposiciones, centros comerciales, hospitales,
o cualquier otro establecimiento de gran tamaño.

La geolocalización indoor además de servir para encontrar localizaciones determinadas, puede ser aplicada para programar 
rutas que se amolden a unos intereses, y así hacer de la experiencia de moverse por nuevos lugares algo más fructífero y gratificante.

\section{Estructura de la memoria}\label{estructura-de-la-memoria}

La memoria tiene la siguiente estructura:

\begin{itemize}
\tightlist
\item
  \textbf{Introducción:} exposición del problema a resolver y cómo lo resuelve el proyecto.
   Estructura de la memoria y listado de materiales
  adjuntos.
\item
  \textbf{Objetivos del proyecto:} objetivos del proyecto.
\item
  \textbf{Conceptos teóricos:} explicación de los conceptos teóricos necesarios para comprender el proyecto.
\item
  \textbf{Técnicas y herramientas:} técnicas y herramientas utilizadas para la realización de proyecto.
\item
  \textbf{Aspectos relevantes del desarrollo:} explicación de dificultades y obstáculos encontrados en la realización del proyecto.
\item
  \textbf{Trabajos relacionados:} trabajos relacionados con el proyecto.
\item
  \textbf{Conclusiones y líneas de trabajo futuras:} conclusiones
  obtenidas tras realizar el proyecto y expectativas de futuro.
\end{itemize}

Junto a la memoria se proporcionan los siguientes anexos:

\begin{itemize}
\tightlist
\item
  \textbf{Plan del proyecto software:} sección centrada planificación temporal y viabilidad del proyecto.
\item
  \textbf{Especificación de requisitos del software:} sección que se centra en la especificación de requisitos.
\item
  \textbf{Especificación de diseño:} sección que se basa en describir la fase de diseño.
\item
  \textbf{Manual del programador:} sección centrada en la explicación del código fuente (estructura, compilación,
  instalación, ejecución, pruebas,etc... ).
\item
  \textbf{Manual de usuario:} sección centrada en la explicación al usuario de como utilizar la herramienta.
\end{itemize}

\section{Materiales adjuntos}\label{materiales-adjuntos}

Los materiales adjuntados con la memoria son: 

\begin{itemize}
\tightlist
\item
	Architect, parte de la herramienta para crear rutas y edificios.
\item
	Viewer, parte de la herramienta encargada de la visualización.
\item	
	Código fuente de la Rest API encargada de interactuar con la BD.
\item	
	Pruebas de integración.
\item	
	Pruebas de estrés y rendimiento.
\end{itemize}

Recursos en internet:

\begin{itemize}
\tightlist
\item
  Architect (Se lanza en el localhost).
\item
  Viewer (Se lanza en el localhost).
\item
  Rest API https://geoindoorapi.herokuapp.com/ .
\item
  Base de datos de firebase  https://geoindoordb.firebaseio.com/ .
\item
  Repositorio del proyecto .
\end{itemize}
