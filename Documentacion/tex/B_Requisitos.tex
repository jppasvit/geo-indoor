\apendice{Especificación de Requisitos}

\section{Introducción}
Este anexo recoge la especificación de requisitos, la cual define el comportamiento del sistema desarrollado, es utilizado tanto por el cliente y por el equipo para tener una concepto generalizado de lo que se quiere, para que ambas partes lleguen a una idea común. Para este fin muchas veces son utilizados diagramas los cuales aclaren el concepto y lo que se necesita.

Podemos distinguir entre requisitos funcionales y requisitos no funcionales.
\begin{itemize}
\item
\textbf{Requisito funcional:} es aquel que especifica una función o un servicio que debe cumplir un sistema.
\item
\textbf{Requisito no funcional:} este especifica restricciones sobre el diseño y la implementción del sistema.
\end{itemize}

Las características de una buena especificación de requisitos son definidas por:
\begin{itemize}
\item
\textbf{Completa:} deben estar todos los requerimientos y todas sus relaciones.
\item
\textbf{Consistente:} todos los requerimientos y otros documentos de especificación se debe relacionar de forma coherente.
\item
\textbf{Inequívoca:} se debe ser claro y darse a entender.
\item
\textbf{Correcta:} el sistema debe cumplir con todos los requisitos especificados.
\item
\textbf{Trazable:} los requerimientos deben de estar ordenados y organizados de una manera tal, que sea sencilla su identificación.
\item
\textbf{Priorizable:} los requisitos deben estar organizados de forma jerárquica, según su importancia.
\item
\textbf{Modificable:} los requerimientos deben ser fácilmente modificables.
\item
\textbf{Verificable:} se debe poder probar todo requerimiento.
\end{itemize}


\section{Objetivos generales}
\begin{itemize}
\item
Desarrollar un sistema que permita la localización del individuo en interiores.
\item
Desarrollar un sistema que permita la localización de lugares en interiores.
\item
Desarrollar un sistema que permita el trazado de rutas en interiores.
\item
Desarrollar un sistema que permita mostrar lugares y rutas predefinidas en interiores.
\end{itemize}
\section{Catalogo de requisitos}
A partir de los objetivos anteriormente citados obtenemos los siguientes requisitos.

\subsection{Requisitos funcionales}

\begin{itemize}
\item
\textbf{RF-1 Gestión de edificios:} la herramienta debe permitir la gestión de un edifico.
\begin{itemize}
	\item
	\textbf{RF-1.1 Emplazar edificio:} la herramienta debe permitir la colocación de un edifico sobre el plano de localización
	\item
	\textbf{RF-1.2 Detallar información del edificio:} se debe permitir adjuntar información al edificio (nombre, descripción, etc\ldots).
	\item
	\textbf{RF-1.3 Emplazar plano:} la herramienta debe permitir la colocación del plano del edifico sobre el plano de localización.
	\item
	\textbf{RF-1.4 Redimensionar y mover el plano:} se debe permitir el movimiento y redimensión del plano añadido.
	\item
	\textbf{RF-1.5 Borrar edifico:} se debe permitir el borrado de un edificio.
\end{itemize}
\item
\textbf{RF-2 Gestión de pois:} la herramienta debe permitir la gestión de pois.
\begin{itemize}
	\item
	\textbf{RF-2.1 Emplazamiento de pois:} se debe permitir la colocación de puntos de interés sobre el plano.
	\item
	\textbf{RF-2.2 Detallar información del poi:} se debe permitir adjuntar información al poi (nombre, descripción, tipo, etc			\ldots).
	\item
	\textbf{RF-2.3 Ubicar poi:} la herramienta debe permitir la colocación del poi sobre una localización del plano de un edifico.
	\item
	\textbf{RF-2.4 Enlazar pois:} se debe permitir crear un camino entre pois.
\end{itemize}
\item
\textbf{RF-3 Gestión de rutas:} se debe permitir la gestión de rutas predefinidas.
\begin{itemize}
	\item
	\textbf{RF-3.1 Crear ruta:} se debe permitir crear una ruta predefinida o predeterminada.
	\item
	\textbf{RF-3.2 Detallar información de la ruta:} se debe permitir agregar información a la ruta, como el nombre o la planta (el número de planta se añade de forma automática, lo que nos permite hacer rutas multinivel).
	\item
	\textbf{RF-3.3 Añadir pois a ruta:} la herramienta debe se capaz de añadir puntos de interés a la ruta.
	\item
	\textbf{RF-3.4 Trazar ruta:} la herramienta tiene que ser capaz de mostrar el trazado de la ruta.
	\item
	\textbf{RF-3.5 Seleccionar ruta:} la herramienta debe permitir elegir entre las diferentes rutas existentes.
	\item
	\textbf{RF-3.6 Limpiar ruta:} la herramienta debe permitir dejar de mostrar el trazado de la ruta.
	\item
	\textbf{RF-3.7 Borrar ruta:} la herramienta debe permitir el borrado de una ruta elegida.
\end{itemize}
\item
\textbf{RF-4 Visualización de edificios:} el edificio debe estar disponible para todos los usuarios si el usuario administrador o creador así lo decide.
\begin{itemize}
	\item
	\textbf{RF-4.1 Buscar edificio:} la herramienta debe permitir que el edificio se pueda encontrar y visualizar a partir de su identificador o nombre.
	\item
	\textbf{RF-4.2 Visualizar información:} la herramienta debe permitir que la información del edificio sea visible a los usuarios visitantes, por lo tanto se debe visualizar también el plano.
\end{itemize}
\item
\textbf{RF-5 Visualización de pois:} los pois de un edificio deben ser visibles
\begin{itemize}
	\item
	\textbf{RF-5.1 Visualizar información:} la herramienta debe permitir que la información del poi sea visible a los usuarios visitantes.
	\item
	\textbf{RF-5.2 Marcar un poi:} la herramienta debe permitir a los usuarios marcar o seleccionar un poi.
	\item
	\textbf{RF-5.3 Proyectar camino de forma automática:} la herramienta debe permitir que al seleccionar un poi se pueda dibujar un camino automático desde la ubicación del usuario hasta el poi indicado.
\end{itemize}
\item
\textbf{RF-6 Visualización de rutas:} la herramienta debe permitir a los usuarios visualizar las rutas predefinidas del edificio seleccionado
\begin{itemize}
	\item
	\textbf{RF-6.1 Seleccionar ruta:} la herramienta debe permitir seleccionar rutas definidas del edificio.
	\item
	\textbf{RF-6.2 Mostrar ruta:} la herramienta debe permitir mostrar el trazado de la ruta.
	\item
	\textbf{RF-6.3 Limpiar ruta:} la herramienta debe permitir dejar de mostrar el trazado de la ruta.
\end{itemize}
\end{itemize}

\subsection{Requisitos no funcionales}

\begin{itemize}
	\item
	\textbf{RF-1 Intuitiva:} la herramienta debe ser intuitiva y fácil de entender para que así el usuario se acostumbre fácilmente a su uso.
	\item
	\textbf{RF-2 Usable:} la herramienta debe ser fácil de usar, para atraer tanto a gente con voluntad de aprender como no.
	\item
	\textbf{RF-3 Portable:} la herramienta debe de necesitar de instalación alguna más allá de sus dependencias lógicas (Python y navegador).
	\item
	\textbf{RF-4 Rápida:} la herramienta debe proporcionar un servicio rápido, y con resultados tempranos para que el usuario este a gusto trabajando desde el primer instante.
	\item
	\textbf{RF-5 Para todo el mundo:} la herramienta debe tener como final, que un gran espectro de la población la use para ser así mas útil para la sociedad.
\end{itemize}

\section{Especificación de requisitos}

\subsection{Actores}

El sistema diferencia entre dos tipos de actores.

\begin{itemize}
	\item
	\textbf{Usuario de Architect:} este usuario debe de estar autenticado para poder trabajar con la herramienta, se encarga de producir información (crear edificios, rutas, pois, etc\ldots) que será vista por los usuarios de Viewer.
	\item
	\textbf{Usuario de Viewer:} este usuario no debe de estar autenticado para poder trabajar con la herramienta, y no se encarga de producir información, si no de su visualización.
\end{itemize}

\subsection{Aclaraciones}

\begin{itemize}
	\item
	\textbf{Geoindoor:} herramienta o aplicación web que esta formada por dos sub aplicaciones, Architect y Viewer.
	\item
	\textbf{Architect:} herramienta o aplicación web que es utilizada por los usuarios para introducir y editar información sobre los edificios, que luego serán disponibles por los usuarios del Viewer.
	\item
	\textbf{Viewer:} herramienta o aplicación web que es utilizada por los usuarios para visualizar la información sobre los edificios (rutas, pois, etc\ldots) introducidos a través del Architect.
\end{itemize}

\newpage
\subsection{Diagrama de casos de uso}

\imagenResize{0.65}{img/GestionDeEdificio}{Gestión de edificio}{GestionDeEdificio}
\newpage
\imagenResize{0.65}{img/GestionDePois}{Gestión de pois}{GestionDePois}
\newpage
\imagenResize{0.55}{img/GestionDeRutas}{Gestión de rutas}{GestionDeRutas}
\newpage
\imagenResize{0.5}{img/VisualizDeEdificios}{Visualización de edificios}{VisualizDeEdificios}
\imagenResize{0.5}{img/VisualizDePois}{Visualización de pois}{VisualizDePois}
\newpage
\imagenResize{0.55}{img/VisualizDeRutas}{Visualización de rutas}{VisualizDeRutas}

\newpage
\subsubsection{Diagrama de caso de uso (General)}

\imagenResize{0.25}{img/DiagramaCasoDeUso}{Diagrama de caso de uso}{DiagramaCasoDeUso}


\newpage
\subsection{Casos de uso}

\tablaSinColores{CU-01 Gestión de edificio}
{L{3.5cm} L{10cm}}
{2}
{Tabla CU-01}
{\textbf{CU-01} & \textbf{Gestión de edificio} \\}
{\textbf{Versión} 				& 1.0\\ 
	\textbf{Autor} 				& Juan Pedro Pascual Vitores\\
	\textbf{Requisitos asociados} 	& RF-1, RF-1.2, RF-1.3, RF-1.4, RF-1.5\\
	\textbf{Descripción} 			& 
	Permite al usuario la administración y gestión de un edifico.\\
	\textbf{Precondiciones} 		& 
	\begin{itemize}
		\item Se encuentra disponible la base de datos.
		\item Se encuentra disponible la REST API.
		\item Se accede como usuario Architect (autenticado).
	\end{itemize}
	\\
	\textbf{Acciones} 				& 
	\begin{enumerate}
		\item El usuario accede a la herramienta Architect.
		\item El usuario se autentica.
		\item El usuario interacciona con la herramienta para introducir la información concerniente a un edificio.
		\item Se introduce la información.
	\end{enumerate}
	\\
	
	\textbf{Postcondiciones} 		& 
	\begin{itemize}
		\item La información se almacena en la base de datos.
	\end{itemize}
	\\
	\textbf{Excepciones} 			& 
	\begin{itemize}
		\item La base de datos no está disponible.
		\item La REST API no está disponible.
		\item La información introducida no es válida
	\end{itemize}
	
	\\
	\textbf{Importancia} 			& Alta\\}

%%%%%%%%%%%%%%%%%%%%%%%%%%%%%%%%%%%%%%%%%%%%%%%%%%%%%%%%%%%%%%%%%%%%%%%%%%%%%%%%%%%%%%%%%%%%%%%%%%%%%%%%%%%%%%%%%%%%%%%%%%%%%

\tablaSinColores{CU-02 Emplazar edificio}
{L{3.5cm} L{10cm}}
{2}
{Tabla CU-02}
{\textbf{CU-02} & \textbf{Emplazar edificio} \\}
{\textbf{Versión} 				& 1.0\\ 
	\textbf{Autor} 				& Juan Pedro Pascual Vitores\\
	\textbf{Requisitos asociados} 	& RF-1.1\\
	\textbf{Descripción} 			& 
	Permite al usuario la colocación de un edifico sobre el plano de localización.\\
	\textbf{Precondiciones} 		& 
	\begin{itemize}
		\item Se encuentra disponible la base de datos.
		\item Se encuentra disponible la REST API.
		\item Se accede como usuario Architect (autenticado).
	\end{itemize}
	\\
	\textbf{Acciones} 				& 
	\begin{enumerate}
		\item El usuario accede a la herramienta Architect.
		\item El usuario se autentica.
		\item El usuario elige una posición en el plano.
		\item El usuario pincha en la pestaña ``Buildings" de su panel de control.
		\item El usuario pincha en el botón ``Add'', después en el icono del edificio y arrastra el puntero hasta una posición del plano.
		\item El usuario pulsa sobre el botón ``Add" del panel emergente, para añadir de forma definitiva el edificio.
	\end{enumerate}
	\\
	
	\textbf{Postcondiciones} 		& 
	\begin{itemize}
		\item La información sobre el edificio se almacena en la base de datos.
		\item La información posteriormente será visible.
	\end{itemize}
	\\
	\textbf{Excepciones} 			& 
	\begin{itemize}
		\item La base de datos no está disponible.
		\item La REST API no está disponible.
		\item La información introducida no es válida
	\end{itemize}
	
	\\
	\textbf{Importancia} 			& Alta\\}

%%%%%%%%%%%%%%%%%%%%%%%%%%%%%%%%%%%%%%%%%%%%%%%%%%%%%%%%%%%%%%%%%%%%%%%%%%%%%%%%%%%%%%%%%%%%%%%%%%%%%%%%%%%%%%%%%%%%%%%%%%%%%

\tablaSinColores{CU-03 Detallar información de edificio}
{L{3.5cm} L{10cm}}
{2}
{Tabla CU-03}
{\textbf{CU-03} & \textbf{Detallar información de edificio} \\}
{\textbf{Versión} 				& 1.0\\ 
	\textbf{Autor} 				& Juan Pedro Pascual Vitores\\
	\textbf{Requisitos asociados} 	& RF-1.2\\
	\textbf{Descripción} 			& 
	Permite al usuario adjuntar información al edificio (nombre, descripción, etc\ldots).\\
	\textbf{Precondiciones} 		& 
	\begin{itemize}
		\item Se encuentra disponible la base de datos.
		\item Se encuentra disponible la REST API.
		\item Se accede como usuario Architect (autenticado).
	\end{itemize}
	\\
	\textbf{Acciones} 				& 
	\begin{enumerate}
		\item El usuario accede a la herramienta Architect.
		\item El usuario se autentica.
		\item El usuario elige una posición en el plano.
		\item El usuario pincha en la pestaña ``Buildings" de su panel de control.
		\item El usuario pincha en el botón ``Add'', después en el icono del edificio y arrastra el puntero hasta una posición del plano.
		\item El usuario rellena el formulario del panel emergente.
		\item El usuario pulsa sobre el botón ``Add" del panel emergente, para añadir de forma definitiva el edificio.
	\end{enumerate}
	\\
	
	\textbf{Postcondiciones} 		& 
	\begin{itemize}
		\item La información sobre el edificio se almacena en la base de datos.
		\item La información posteriormente será visible.
	\end{itemize}
	\\
	\textbf{Excepciones} 			& 
	\begin{itemize}
		\item La base de datos no está disponible.
		\item La REST API no está disponible.
		\item La información introducida no es válida
	\end{itemize}
	
	\\
	\textbf{Importancia} 			& Alta\\}

%%%%%%%%%%%%%%%%%%%%%%%%%%%%%%%%%%%%%%%%%%%%%%%%%%%%%%%%%%%%%%%%%%%%%%%%%%%%%%%%%%%%%%%%%%%%%%%%%%%%%%%%%%%%%%%%%%%%%%%%%%%%%

\tablaSinColores{CU-04 Emplazar plano}
{L{3.5cm} L{10cm}}
{2}
{Tabla CU-04}
{\textbf{CU-04} & \textbf{Emplazar plano} \\}
{\textbf{Versión} 				& 1.0\\ 
	\textbf{Autor} 				& Juan Pedro Pascual Vitores\\
	\textbf{Requisitos asociados} 	& RF-1.3\\
	\textbf{Descripción} 			& 
	Permite al usuario la colocación del plano del edificio sobre el plano de localización.\\
	\textbf{Precondiciones} 		& 
	\begin{itemize}
		\item Se encuentra disponible la base de datos.
		\item Se encuentra disponible la REST API.
		\item Se tiene una imagen del plano o un pdf del mismo.
		\item Se accede como usuario Architect (autenticado).
	\end{itemize}
	\\
	\textbf{Acciones} 				& 
	\begin{enumerate}
		\item El usuario accede a la herramienta Architect.
		\item El usuario se autentica.
		\item El usuario pincha sobre el icono de edificio que aparece en el plano de localización, si no, debe emplazar un edificio.
		\item El usuario pincha en la pestaña ``Floors" de su panel de control.
		\item El usuario pincha en el text-area de ``Floor Number'', para introducir a que planta pertenece el plano.
		\item El usuario pincha en el botón de ``Floor Plan''.
		\item El usuario elige el plano a introducir.
		\item El usuario pulsa sobre el botón \checkmark para emplazar definitivamente el plano.
	\end{enumerate}
	\\
	
	\textbf{Postcondiciones} 		& 
	\begin{itemize}
		\item El plano del edificio se almacena en la base de datos.
		\item La información posteriormente será visible.
	\end{itemize}
	\\
	\textbf{Excepciones} 			& 
	\begin{itemize}
		\item La base de datos no está disponible.
		\item La REST API no está disponible.
		\item La información introducida no es válida
	\end{itemize}
	
	\\
	\textbf{Importancia} 			& Alta\\}

%%%%%%%%%%%%%%%%%%%%%%%%%%%%%%%%%%%%%%%%%%%%%%%%%%%%%%%%%%%%%%%%%%%%%%%%%%%%%%%%%%%%%%%%%%%%%%%%%%%%%%%%%%%%%%%%%%%%%%%%%%%%%

\tablaSinColores{CU-05 Redimensionar y mover plano}
{L{3.5cm} L{10cm}}
{2}
{Tabla CU-05}
{\textbf{CU-05} & \textbf{Redimensionar y mover plano} \\}
{\textbf{Versión} 				& 1.0\\ 
	\textbf{Autor} 				& Juan Pedro Pascual Vitores\\
	\textbf{Requisitos asociados} 	& RF-1.4\\
	\textbf{Descripción} 			& 
	Permite al usuario el movimiento y redimensión del plano añadido.\\
	\textbf{Precondiciones} 		& 
	\begin{itemize}
		\item Se encuentra disponible la base de datos.
		\item Se encuentra disponible la REST API.
		\item Se tiene una imagen del plano o un pdf del mismo.
		\item Se accede como usuario Architect (autenticado).
	\end{itemize}
	\\
	\textbf{Acciones} 				& 
	\begin{enumerate}
		\item El usuario accede a la herramienta Architect.
		\item El usuario se autentica.
		\item El usuario pincha sobre el icono de edificio que aparece en el plano de localización, si no, debe emplazar un edificio.
		\item El usuario pincha en la pestaña ``Floors" de su panel de control.
		\item El usuario pincha en el text-area de ``Floor Number'', para introducir a que planta pertenece el plano.
		\item El usuario pincha en el botón de ``Floor Plan''.
		\item El usuario elige el plano a introducir.
		\item El usuario mueve y redimensiona la imagen a través de los iconos típicos de redimensión y giro de imagen.
		\item El usuario pulsa sobre el botón \checkmark para emplazar definitivamente el plano.
	\end{enumerate}
	\\
	
	\textbf{Postcondiciones} 		& 
	\begin{itemize}
		\item El plano del edificio se almacena en la base de datos.
		\item La información posteriormente será visible.
	\end{itemize}
	\\
	\textbf{Excepciones} 			& 
	\begin{itemize}
		\item La base de datos no está disponible.
		\item La REST API no está disponible.
		\item La información introducida no es válida
	\end{itemize}
	
	\\
	\textbf{Importancia} 			& Alta\\}

%%%%%%%%%%%%%%%%%%%%%%%%%%%%%%%%%%%%%%%%%%%%%%%%%%%%%%%%%%%%%%%%%%%%%%%%%%%%%%%%%%%%%%%%%%%%%%%%%%%%%%%%%%%%%%%%%%%%%%%%%%%%%

\tablaSinColores{CU-06 Borrar edificio}
{L{3.5cm} L{10cm}}
{2}
{Tabla CU-06}
{\textbf{CU-06} & \textbf{Borrar edificio} \\}
{\textbf{Versión} 				& 1.0\\ 
	\textbf{Autor} 				& Juan Pedro Pascual Vitores\\
	\textbf{Requisitos asociados} 	& RF-1.5\\
	\textbf{Descripción} 			& 
	Permite al usuario el borrado de un edificio.\\
	\textbf{Precondiciones} 		& 
	\begin{itemize}
		\item Se encuentra disponible la base de datos.
		\item Se encuentra disponible la REST API.
		\item Se accede como usuario Architect (autenticado).
	\end{itemize}
	\\
	\textbf{Acciones} 				& 
	\begin{enumerate}
		\item El usuario accede a la herramienta Architect.
		\item El usuario se autentica.
		\item El usuario pincha sobre el icono de edificio que aparece en el plano de localización.
		\item El usuario pincha en la pestaña ``Buildings" de su panel de control.
		\item El usuario pincha en el botón ``Delete'', y posteriormente en el botón ``Confirm Deletion" para borrar de forma definitiva el edificio.
	\end{enumerate}
	\\
	
	\textbf{Postcondiciones} 		& 
	\begin{itemize}
		\item El plano del edificio se almacena en la base de datos.
		\item La información del edificio posteriormente no será visible.
	\end{itemize}
	\\
	\textbf{Excepciones} 			& 
	\begin{itemize}
		\item La base de datos no está disponible.
		\item La REST API no está disponible.
	\end{itemize}
	
	\\
	\textbf{Importancia} 			& Alta\\}

%%%%%%%%%%%%%%%%%%%%%%%%%%%%%%%%%%%%%%%%%%%%%%%%%%%%%%%%%%%%%%%%%%%%%%%%%%%%%%%%%%%%%%%%%%%%%%%%%%%%%%%%%%%%%%%%%%%%%%%%%%%%%

\tablaSinColores{CU-07 Gestión de pois}
{L{3.5cm} L{10cm}}
{2}
{Tabla CU-07}
{\textbf{CU-07} & \textbf{Gestión de pois} \\}
{\textbf{Versión} 				& 1.0\\ 
	\textbf{Autor} 				& Juan Pedro Pascual Vitores\\
	\textbf{Requisitos asociados} 	& RF-2, RF-2.1, RF-2.2, RF-2.3, RF-2.4\\
	\textbf{Descripción} 			& 
	Permite al usuario la gestión y administración de pois.\\
	\textbf{Precondiciones} 		& 
	\begin{itemize}
		\item Se encuentra disponible la base de datos.
		\item Se encuentra disponible la REST API.
		\item Se accede como usuario Architect (autenticado).
	\end{itemize}
	\\
	\textbf{Acciones} 				& 
	\begin{enumerate}
		\item El usuario accede a la herramienta Architect.
		\item El usuario se autentica.
		\item El usuario pincha sobre el icono de edificio que aparece en el plano de localización, si no, debe emplazar un edificio y su respectivo plano.
		\item El usuario pincha en la pestaña ``POIs" de su panel de control.
		\item El usuario interacciona con la herramienta para introducir la información concerniente a los pois de un edificio.
	\end{enumerate}
	\\
	
	\textbf{Postcondiciones} 		& 
	\begin{itemize}
		\item El plano del edificio se almacena en la base de datos.
		\item La información de los pois del edificio será visible.
	\end{itemize}
	\\
	\textbf{Excepciones} 			& 
	\begin{itemize}
		\item La base de datos no está disponible.
		\item La REST API no está disponible.
		\item No hay edificio donde ubicar los pois.
	\end{itemize}
	
	\\
	\textbf{Importancia} 			& Alta\\}

%%%%%%%%%%%%%%%%%%%%%%%%%%%%%%%%%%%%%%%%%%%%%%%%%%%%%%%%%%%%%%%%%%%%%%%%%%%%%%%%%%%%%%%%%%%%%%%%%%%%%%%%%%%%%%%%%%%%%%%%%%%%%