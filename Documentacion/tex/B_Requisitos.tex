\apendice{Especificación de Requisitos}

\section{Introducción}
EL anexo recoge la especifcación de requisitos que define el comportamiento del sistema desarrollado, y sirve como documento contractual entre el cliente y el equipo así como la documentación correspondiente al análisis a la aplicación.

Las características de una buena Especificación de requisitos son definidas por:
\begin{itemize}
\item
Completa: Todos los requerimientos deben estar reflejados en ella y todas las referencias deben estar definidas.
\item
Consistente: Debe ser coherente con los propios requerimientos y también con otros documentos de especificación.
\item
Inequívoca: La redacción debe ser clara de modo que no se pueda mal interpretar.
\item
Correcta: El software debe cumplir con los requisitos de la especificación.
\item
Trazable: Se refiere a la posibilidad de verificar la historia, ubicación o aplicación de un ítem a través de su identificación almacenada y documentada.
\item
Priorizable: Los requisitos deben poder organizarse jerárquicamente según su relevancia para el negocio y clasificándolos en esenciales, condicionales y opcionales.
\item
Modificable: Aunque todo requerimiento es modificable, se refiere a que debe ser fácilmente modificable.
\item
Verificable: Debe existir un método finito sin costo para poder probarlo
\end{itemize}


\section{Objetivos generales}
\begin{itemize}
\item
Desarrollar un sistema que con múltiples aplicaciones permita añadir
servicios de geolocalización dentro de edificios.
\item
Facilitar la localización de lugares dentro de edificios.
\item
Creación de rutas para determinados entornos.
\end{itemize}
\section{Catalogo de requisitos}

\section{Especificación de requisitos}


