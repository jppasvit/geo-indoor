\apendice{Especificación de Requisitos}

\section{Introducción}
Este anexo recoge la especificación de requisitos, la cual define el comportamiento del sistema desarrollado, es utilizado tanto por el cliente y por el equipo para tener una concepto generalizado de lo que se quiere, para que ambas partes lleguen a una idea común. Para este fin muchas veces son utilizados diagramas los cuales aclaren el concepto y lo que se necesita.

Podemos distinguir entre requisitos funcionales y requisitos no funcionales.
\begin{itemize}
\item
\textbf{Requisito funcional:} es aquel que especifica una función o un servicio que debe cumplir un sistema.
\item
\textbf{Requisito no funcional:} este especifica restricciones sobre el diseño y la implementción del sistema.
\end{itemize}

Las características de una buena especificación de requisitos son definidas por:
\begin{itemize}
\item
\textbf{Completa:} deben estar todos los requerimientos y todas sus relaciones.
\item
\textbf{Consistente:} todos los requerimientos y otros documentos de especificación se debe relacionar de forma coherente.
\item
\textbf{Inequívoca:} se debe ser claro y darse a entender.
\item
\textbf{Correcta:} el sistema debe cumplir con todos los requisitos especificados.
\item
\textbf{Trazable:} los requerimientos deben de estar ordenados y organizados de una manera tal, que sea sencilla su identificación.
\item
\textbf{Priorizable:} los requisitos deben estar organizados de forma jerárquica, según su importancia.
\item
\textbf{Modificable:} los requerimientos deben ser fácilmente modificables.
\item
\textbf{Verificable:} se debe poder probar todo requerimiento.
\end{itemize}


\section{Objetivos generales}
\begin{itemize}
\item
Desarrollar un sistema que permita la localización del individuo en interiores.
\item
Desarrollar un sistema que permita la localización de lugares en interiores.
\item
Desarrollar un sistema que permita el trazado de rutas en interiores.
\item
Desarrollar un sistema que permita mostrar lugares y rutas predefinidas en interiores.
\end{itemize}
\section{Catalogo de requisitos}
A partir de los objetivos anteriormente citados obtenemos los siguientes requisitos.

\subsection{Requisitos Funcionales}

\begin{itemize}
\item
\textbf{RF-1 Gestión de edificio:} la herramienta debe permitir la gestión de un edifico.
\begin{itemize}
	\item
	\textbf{RF-1.1 Emplazamiento de edificio:} la herramienta debe permitir la colocación de un edifico sobre el plano de localización
	\item
	\textbf{RF-1.2 Detallar información del plano:} se debe permitir adjuntar información al edificio (nombre, descripción, etc			\ldots).
	\item
	\textbf{RF-1.3 Emplazamiento de plano:} la herramienta debe permitir la colocación del plano del edifico sobre el plano de 			localización.
	\item
	\textbf{RF-1.4 Redimensión y movimiento del plano:} se debe permitir el movimiento y redimensión del plano añadido.
\end{itemize}
\item
\textbf{RF-2 Gestión de pois:} la herramienta debe permitir la gestión de pois.
\begin{itemize}
	\item
	\textbf{RF-2.1 Emplazamiento de pois:} se debe permitir la colocación de puntos de interés sobre el plano.
	\item
	\textbf{RF-2.2 Detallar información del poi:} se debe permitir adjuntar información al poi (nombre, descripción, tipo, etc			\ldots).
	\item
	\textbf{RF-2.3 Ubicación del poi:} la herramienta debe permitir la colocación del poi sobre una localización del plano de un edifico.
\end{itemize}
\item
\textbf{RF-3 Enlazado de pois:} se debe permitir crear un camino entre pois.
\item
\textbf{RF-4 Gestión de rutas:} se debe permitir la gestión de rutas predefinidas.
\item
\textbf{RF-:}
\item
\textbf{RF-:}
\item
\textbf{RF-:}
\end{itemize}

\section{Especificación de requisitos}


