\capitulo{5}{Aspectos relevantes del desarrollo del proyecto}

En el desarrollo de este proyecto han surgido varios obstáculos que han sido resueltos de diversas maneras. A continuación los explicaremos.

\section{Análisis e investigación de la aplicación }\label{analisis-investigacion}

El principal problema que me encontré fue entender la aplicación sobre la que trabajar y como hacerlo, el principal problema que tuve es que no había una verdadera documentación donde se explicara como funciona al aplicación y cuales son sus funciones y métodos. Con lo cual tuve que investigar y analizar la aplicación a conciencia hasta entenderla.

Otro gran problema que surgió fue entender y crear el servidor que ofrecían, hasta que se decidió trabajar con la aplicación por separado, es decir Architect por un lado y Viewer por otro y después unificarlos para crear el servidor. 

\section{Viewer}\label{viewer}

El problema que surgió con ello, fue a  la hora de crear la función que recoge la posición del usuario para compararla con los puntos determinados. El problema surgió porque en la función getPointLocationUser, watchPosition además de darte las coordenadas del usuario en le momento también te bloquea la la muestra de la posicion del usuario (el punto azul que indica donde esta el usuario), eso es debido a que watchPosition tiene una función callback que bloqueaba. Se consiguió resolver usando clearWatch, para evitar el callback.

\section{Heroku}\label{Heroku}

El principal problema que tuve con ello es que no lo conocía, por lo que tuve que adaptarme y aprender su funcionamiento, aunque hay que decir que los primeros pasos los facilitan gracias a sus tutoriales.

Una vez que se aprendió como funcionaba heroku se comenzo a realizar la Rest API cuyo principal problema fue entender como funcionaba Node.js y las llamadas POST Y GET en Node.js con express.

También es importante decir que con las llamadas al servidor surgieron problemas CORS (access-control-allow-origin),
que se ha solucionado añadiendo las cabeceras de Access-Control-Allow-Origin. 

\section{Firebase}\label{Firebase}

Lo que sucedió con Firebase esta relacionado con su base de datos JSON, cuya estructura de base de datos tuve que idear para que se pudiese conseguir una gran cantidad de información de calidad evitando cantidad de información de poco valor.

La idea es utilizar keys que iguales y añadir una palabra clave para utilizar esa clave como otra key. Es decir la key id2 es la key de un edificio y la key id2rutas tiene las rutas del edificio de key id2.


\section{Architect}\label{Architect} 

El principal problema en la parte de la aplicación de creación de caminos,edificios, y pois, es que las llamadas a al servidor de AnyPlace de determinas url daba problemas de CORS, lo que se soluciono obteniendo información a través de AngularJS 


\section{Otros problemas}\label{otros}

Existieron otros muchos problemas pero que no son comentables ya que no son de tanto interés como estos.

Pero si hay que comentar alguno problema más que me he encontrado, ha sido el uso de jquery para las llamadas, sobre todo con el dataType, el hosting de firebase que no permite hacer llamadas fuera de su dominio y los dynos de heroku.