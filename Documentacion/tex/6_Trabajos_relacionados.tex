\capitulo{6}{Trabajos relacionados}

Los trabajos que estan relacionados con este son \href{https://www.google.es/maps}{Google Maps}, \href{https://anyplace.cs.ucy.ac.cy/}{AnyPlace} y \href{https://www.goindoor.co/}{goindoor}.

\section{Google Maps}\label{GoogleMaps}

Google Maps es la semilla del proyecto ya que AnyPlace se basa en ello.
Google Maps ofrece un API en JavaScript que ha sido ampliamente utilizado en este proyecto, la principal ventaja que tiene utilizar el API de Google Maps es el soporte y la atención, además la amplia comunidad que trabaja con Google Maps hace que se resuelvan la dudas fácilmente. 

La principal diferencia ente Google Maps y Geoindoor es que una Google Maps busca la geolocalización en extensiones amplias de terreno mientras que Geoindoor la busca en interirores, aunque sigue marcando rutas en largas distancias también.

\section{AnyPlace}\label{AnyPlace}

AnyPlace es la aplicación en la que se basa Geoindoor, con lo que las principales diferencias entre AnyPlace y Geoindoor son que Geoindoor además de cambios de estilo ofrece el agregado de rutas al edificio y su interactividad.

\section{goindoor}\label{goindoor}

goindoor es otra sistema utilizado para la geolocalización indoor y la principal diferencia entre Geoindoor y goindoor es que, goindoor utiliza bacons, o anclajes que son utilizados de sensores para obtener la geolocalización.
