\capitulo{6}{Trabajos relacionados}

Los trabajos que estan relacionados con este son \href{https://www.google.es/maps}{Google Maps}, \href{https://anyplace.cs.ucy.ac.cy/}{AnyPlace} y \href{https://www.goindoor.co/}{goindoor}.

\section{Google Maps}\label{GoogleMaps}

Google Maps es la semilla del proyecto ya que AnyPlace se basa en ello.
Google Maps ofrece un API en JavaScript que ha sido ampliamente utilizada en este proyecto, la principal ventaja que tiene utilizar el API de Google Maps es el soporte y la atención, además la comunidad que trabaja con Google Maps hace que se resuelvan la dudas fácilmente. 

Hay que reconocer que la documentación que ofrece Google Maps ha sido de gran ayuda, así como su sistema de dibujado sobre el plano. 

La principal diferencia ente Google Maps y Geoindoor es que Google Maps busca la geolocalización en extensiones amplias de terreno mientras que Geoindoor la busca en interiores, aunque sigue marcando rutas en largas distancias también. Otra diferencia es que Google Maps se centra en crear rutas automáticas para el usuario, mientras que Geoindoor busca crear rutas predeterminadas creadas por usuarios para usuarios.


\section{AnyPlace}\label{AnyPlace}

AnyPlace es la aplicación en la que se basa Geoindoor. AnyPlace siguiendo un poco la línea de Google Maps, busca automatizar las rutas que uno debe seguir, mientras que Geoindoor ofrece rutas predeterminadas creadas por un usuario administrador para que las siga otro usuario cliente, de esta manera se mejora la experiencia de la geolocalización en interiores y permite a los usuarios administradores que los usuarios clientes visiten o localicen los lugares de una forma determinada.

\section{Goindoor}\label{goindoor}

Goindoor es otra sistema utilizado para la geolocalización indoor y la principal diferencia entre Geoindoor y goindoor es que, Goindoor utiliza bacons, o anclajes que son utilizados de sensores para obtener la geolocalización, mientras que Geoindoor utiliza el sistema de geolocalización de Google Maps, Goindor tampoco tiene un sistema de rutas predeterminadas.
